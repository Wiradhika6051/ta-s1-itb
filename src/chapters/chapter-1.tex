\chapter{Pendahuluan}

\section{Latar Belakang}
\label{sec:latarbelakang}
[\hl{TODO: UBAH BIAR LEBIH SESUAI DENGAN ABSTRAK YANG SEKARANG (DULU FOKUS KE NGEBUAT ILENYA, SKRG FOKUS KE NGEBUAT ILENYA JADI VISUALISASI EKSEKUSI KODE)}]
Pesatnya perkembangan teknologi digital membuka potensi yang luas dalam pembelajaran secara daring. Dengan bantuan internet, pembelajaran dapat dilakukan di mana saja dan kapan saja. Pembelajaran tidak lagi terkekang batas geografis dan waktu sehingga keterhubungan dan interaksi antara pengajar dan pelajar lebih banyak \parencite{choy2004interactive,keengwe2010towards,psotka2012ile}.

Namun, interaksi antara pelajar dan pengajar pada pembelajaran daring tidak dapat disamakan dengan pembelajaran tradisional. Interaksi seperti melemparkan pertanyaan secara lisan kepada siswa, meminta siswa memecahkan persoalan di depan kelas, dan hal-hal serupa tidak dapat dilakukan dalam pembelajaran daring yang asinkron karena hubungan antara pelajar dan pengajar tidak dilakukan secara langsung. Hal ini menimbulkan adanya pro dan kontra terkait pembelajaran secara daring \parencite{radovic2010advantages}.

Seiring dengan perkembangan interaktivitas antara manusia dan komputer, media pembelajaran secara daring semakin terbuka dengan adanya \textit{interactive learning environment} (ILE) yang bermunculan \parencite{psotka2012ile}. Bahkan, ILE berpotensi dapat melebihi kemampuan pembelajaran tradisional dengan adanya integrasi dengan teknologi baik perangkat lunak hingga perangkat keras khusus sehingga dapat mendukung pembelajaran yang mulanya tidak dapat atau sulit dilaksanakan secara luring. Salah satu bidang pembelajaran yang berdampak besar dari adanya ILE ini adalah pemrograman.

ILE memungkinkan pembelajaran pemrograman lebih efektif dan efisien. Seperti yang dijelaskan oleh \textcite{choy2004interactive}, teknologi dapat dimanfaatkan untuk meningkatkan kinerja pembelajaran asinkron sehingga pelatihan pemrograman yang biasanya dilaksanakan dengan lokakarya secara sinkron antara pelajar dan pengajar, kini dapat dilakukan secara asinkron sehingga memungkinkan pembelajaran yang terdistribusi dan latar belakang pelajar yang heterogen.

Agar dapat mendukung pembelajaran secara daring, dibutuhkan 4 komponen sistem menurut \textcite{choy2004interactive}, yaitu lingkungan yang mendukung aktivitas pemrograman secara praktis, interaksi pelajar dan pengajar secara asinkron, analisis dari hasil pekerjaan pelajar, serta manajemen distribusi dan pengumpulan tugas.

Dalam tugas akhir ini, akan dibahas mengenai implementasi ILE yang dapat mendukung aktivitas pemrograman secara praktis beserta ragam implementasi pada platform yang sudah ada, yang kemudian akan dipilih, dimodifikasi, dan diimplementasikan sesuai dengan kebutuhan pada platform pembelajaran pemrograman KodeBareng.

% Pada zaman ini, teknologi berkembang dengan pesat. Dengan adanya teknologi yang mumpuni untuk melakukan pembelajaran secara daring, akses untuk melakukan pembelajaran secara daring sudah terbuka lebar. Sudah banyak platform pembelajaran daring yang terdapat di internet dengan berbagai macam bidang ilmu. Media pembelajarannya juga beragam, mulai dari video, seminar web, pelatihan daring melalui konferensi video, dll. Pelajar kemudian dapat mengetes materi yang telah dia dapat dengan mengerjakan soal-soal berupa latihan dan kuis.

% Khususnya dalam pembelajaran pemrograman, dalam pengerjaan latihan diperlukan alat yang dapat mengetes kode hasil buatan pelajar tersebut dan mengevaluasi kebenarannya. Alat ini biasa disebut sebagai sistem penilaian otomatis atau \textit{automatic grading}. Pelajar dapat membuat kodenya terlebih dahulu, lalu menyerahkan hasil pekerjaannya ke sistem untuk dinilai. Sistem ini sering dipakai oleh berbagai macam lembaga dan institusi dalam pembelajaran pemrograman. Namun, hal ini dapat membuat masalah-masalah yang tidak relevan dalam pengerjaan latihan seperti kesalahan dalam melakukan instalasi, perangkat yang kurang mendukung, perbedaan versi instalasi, dan permasalahan lainnya yang dapat menghambat proses pembelajaran.

% Maka dari itu, diperlukan sistem \textit{environment development} pada platform pembelajaran yang dapat digunakan untuk melakukan eksekusi kode sehingga pengguna dapat langsung mencoba memprogram melalui web dan melihat hasil eksekusi programnya beserta penilaian terkait apabila digunakan dalam pengerjaan latihan soal. Dengan adanya sistem ini, pengguna tidak perlu melakukan instalasi apapun saat mengerjakan latihan soal sehingga diharapkan dapat meningkatkan kenyamanan dan efektivitas pembelajaran..

\section{Rumusan Masalah}
Berdasarkan latar belakang yang telah dijelaskan sebelumnya, rumusan masalah tugas akhir ini adalah:

\begin{enumerate}
      \item Apa kebutuhan yang diperlukan dalam ILE agar dapat mendukung pembelajaran pemrograman secara daring?
      \item Bagaimana rancangan dan konstruksi sistem ILE yang dapat memenuhi kebutuhan pengguna tersebut?
      \item Bagaimana kinerja sistem ILE yang telah diimplementasikan baik dari sisi sistem maupun dari sisi pengguna?
\end{enumerate}

\section{Tujuan}
Tujuan dari tugas akhir ini adalah untuk membuat ILE yang dapat membantu pemula yang ingin belajar pemrograman secara daring sehingga proses pembelajarannya lebih efektif dan efisien.

\section{Batasan Masalah}
Dalam penelitian ini, terdapat batasan masalah yang perlu didefinisikan agar sesuai lingkup pengerjaannya. Batasan masalah untuk tugas akhir ini yaitu sistem dibangun pada web platform pembelajaran pemrograman \href{https://kodebareng.id}{KodeBareng}.

\section{Metodologi}
Metodologi yang digunakan pada pengerjaan tugas akhir ini adalah sebagai berikut:

\begin{enumerate}
      \item Analisis Masalah dan Pemilihan Pendekatan \\
            Dari rumusan masalah yang telah ditentukan, dilakukan pemilihan pendekatan yang sesuai untuk memecahkan masalah tersebut. Pemilihan pendekatan dilakukan dengan perbandingan implementasi sistem yang sudah ada dan melalui studi literatur. Pada tahap ini juga dilakukan analisis kriteria dan kebutuhan sistem yang sesuai dengan kebutuhan pengguna.
      \item Perancangan Desain dan Konstruksi \\
            Setelah pendekatan dan kebutuhan sistem telah ditentukan, dilakukan perancangan solusi. Perancangan solusi terdiri dari perancangan infrastruktur dan perancangan program. Perancangan infrastruktur terdiri dari pemilihan infrastruktur yang digunakan untuk menjalankan program. Perancangan program terdiri dari perancangan \textit{frontend} serta \textit{backend} dari sistem yang dibuat.
      \item Implementasi Solusi \\
            Pada tahap ini, solusi yang telah didesain diimplementasikan dan diintegrasikan dengan platform pembelajaran pemrograman \href{https://kodebareng.id}{Kodebareng}.
      \item Pengujian dan Validasi \\
            Pada tahap ini, dilakukan pengujian terhadap sistem yang sudah diimplementasikan. Aspek yang akan diuji adalah efektivitas dari sisi pengguna serta kinerja sistem selama melayani pengguna.
\end{enumerate}

% \section{Jadwal Pelaksanaan Tugas Akhir}
% Berikut adalah jadwal pelaksanaan tugas akhir I dan II per minggu.

% \begin{longtable}{ |c| >{\setlength{\baselineskip}{0.75\baselineskip}}p{0.17\linewidth} |c|c|c|c|c|c|c|c|c|c|c|c|c|c|c|c| }
%     \caption{\label{tab:jadwal-pelaksaan-ta1}Jadwal Pelaksanaan Tugas Akhir I}                                                                                                                                                                                                                                                                                                                                                                                                                   \\
%     \hline
%     \rowcolor{gray!30}
%                           &                                                       & \multicolumn{2}{c|}{Sep} & \multicolumn{4}{c|}{Okt} & \multicolumn{4}{c|}{Nov} & \multicolumn{4}{c|}{Des} & \multicolumn{2}{c|}{Jan}                                                                                                                                                                                                                                                                         \\
%     \cline{3-18}
%     \rowcolor{gray!30}
%     \multirow{-2}{*}{No.} & \multirow{-2}{0.17\linewidth}{Milestone/ Deliverable} & 3                        & 4                        & 1                        & 2                        & 3                        & 4                     & 1                     & 2                     & 3                     & 4                     & 1                     & 2                     & 3                     & 4                     & 1                     & 2                     \\
%     \hline
%     1                     & Studi Literatur                                       & \cellcolor{yellow!60}    & \cellcolor{yellow!60}    & \cellcolor{yellow!60}    & \cellcolor{yellow!60}    & \cellcolor{yellow!60}    &                       &                       &                       &                       &                       &                       &                       &                       &                       &                       &                       \\
%     \hline
%     2                     & Pendahuluan                                           &                          &                          &                          &                          & \cellcolor{yellow!60}    & \cellcolor{yellow!60} & \cellcolor{yellow!60} & \cellcolor{yellow!60} &                       &                       &                       &                       &                       &                       &                       &                       \\
%     \hline
%     3                     & Rencana Penyelesaian Masalah                          &                          &                          &                          &                          &                          &                       &                       & \cellcolor{yellow!60} & \cellcolor{yellow!60} & \cellcolor{yellow!60} & \cellcolor{yellow!60} & \cellcolor{yellow!60} &                       &                       &                       &                       \\
%     \hline
%     4                     & Buku TA 1                                             &                          &                          &                          &                          &                          &                       &                       &                       &                       &                       &                       & \cellcolor{yellow!60} & \cellcolor{yellow!60} & \cellcolor{yellow!60} &                       &                       \\
%     \hline
%     5                     & Sidang TA 1                                           &                          &                          &                          &                          &                          &                       &                       &                       &                       &                       &                       &                       &                       &                       & \cellcolor{yellow!60} & \cellcolor{yellow!60} \\
%     \hline
% \end{longtable}

% \setlength{\tabcolsep}{5.5pt}
% \begin{longtable}{ |c| >{\setlength{\baselineskip}{0.75\baselineskip}}p{0.17\linewidth} |c|c|c|c|c|c|c|c|c|c|c|c|c|c|c|c|c| }
%     \caption{\label{tab:jadwal-pelaksaan-ta2}Jadwal Pelaksanaan Tugas Akhir II}                                                                                                                                                                                                                                                                                                                                                                                                                                          \\
%     \hline
%     \rowcolor{gray!30}
%                           &                                                       & \multicolumn{2}{c|}{Jan} & \multicolumn{4}{c|}{Feb} & \multicolumn{5}{c|}{Mar} & \multicolumn{4}{c|}{Apr} & \multicolumn{2}{c|}{Mei}                                                                                                                                                                                                                                                                                                 \\
%     \cline{3-19}
%     \rowcolor{gray!30}
%     \multirow{-2}{*}{No.} & \multirow{-2}{0.17\linewidth}{Milestone/ Deliverable} & 3                        & 4                        & 1                        & 2                        & 3                        & 4                     & 1                     & 2                     & 3                     & 4                     & 5                     & 1                     & 2                     & 3                     & 4                     & 1                     & 2                     \\
%     \hline
%     1                     & Analisis dan Perancangan Sistem                       & \cellcolor{yellow!60}    & \cellcolor{yellow!60}    & \cellcolor{yellow!60}    &                          &                          &                       &                       &                       &                       &                       &                       &                       &                       &                       &                       &                       &                       \\
%     \hline
%     2                     & Implementasi dan Pengujian                            &                          & \cellcolor{yellow!60}    & \cellcolor{yellow!60}    & \cellcolor{yellow!60}    & \cellcolor{yellow!60}    & \cellcolor{yellow!60} & \cellcolor{yellow!60} & \cellcolor{yellow!60} & \cellcolor{yellow!60} & \cellcolor{yellow!60} & \cellcolor{yellow!60} &                       &                       &                       &                       &                       &                       \\
%     \hline
%     3                     & Kesimpulan dan Saran                                  &                          &                          &                          &                          &                          &                       &                       &                       & \cellcolor{yellow!60} & \cellcolor{yellow!60} & \cellcolor{yellow!60} & \cellcolor{yellow!60} & \cellcolor{yellow!60} &                       &                       &                       &                       \\
%     \hline
%     4                     & Buku TA II                                            &                          &                          &                          &                          &                          &                       &                       &                       &                       &                       &                       &                       & \cellcolor{yellow!60} & \cellcolor{yellow!60} & \cellcolor{yellow!60} &                       &                       \\
%     \hline
%     5                     & Sidang TA II                                          &                          &                          &                          &                          &                          &                       &                       &                       &                       &                       &                       &                       &                       &                       &                       & \cellcolor{yellow!60} & \cellcolor{yellow!60} \\
%     \hline
% \end{longtable}
% \setlength{\tabcolsep}{6pt}

\section{Sistematika Pembahasan}
\blindtext