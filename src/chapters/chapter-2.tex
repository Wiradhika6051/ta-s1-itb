\chapter{Studi Literatur}

Bab Studi Literatur digunakan untuk mendeskripsikan kajian literatur yang terkait dengan persoalan tugas akhir. Tujuan studi literatur adalah:

\begin{enumerate}
    \item menunjukkan kepada pembaca adanya gap seperti pada rumusan masalah yang memang belum terselesaikan,
    \item memberikan pemahaman yang secukupnya kepada pembaca tentang teori atau pekerjaan terkait yang terkait langsung dengan penyelesaian persoalan, serta
    \item menyampaikan informasi apa saja yang sudah ditulis/dilaporkan oleh pihak lain (peneliti/Tugas Akhir/Tesis) tentang hasil penelitian/pekerjaan mereka yang sama atau mirip kaitannya dengan persoalan tugas akhir.
\end{enumerate}

\section{Pembelajaran}
\par
Belajar adalah perubahan yang relatif permanen dalam perilaku atau potensi perilaku sebagai hasil dari pengalaman atau latihan yang diperkuat \parencite{slavin2017}. Belajar merupakan akibat adanya interaksi antara stimulus dan respons. Seseorang dianggap telah belajar sesuatu jika dia dapat menunjukkan perubahan perilakunya. Menurut teori ini, dalam belajar yang penting adalah input yang berupa stimulus dan output yang berupa respons.

\par
Salah satu pengertian pembelajararan dikemukakan oleh \textcite{gagne1970} yaitu pembelajaran adalah seperangkat peristiwa -peristiwa eksternal yang dirancang untuk mendukung beberapa proses belajar yang bersifat internal. Lebih lanjut, Gagne (1985) mengemukakan teorinya lebih lengkap dengan mengatakan bahwa pembelajaran dimaksudkan untuk menghasilkan belajar, situasi eksternal harus dirancang sedemikian rupa untuk mengaktifkan, mendukung, dan mempertahankan proses internal yang terdapat dalam setiap peristiwa belajar. 

\section{Pemrograman}
\blindtext

\section{Pembelajaran Pemrograman secara Daring}
\blindtext

\section{\textit{Interactive Coding}}
\blindtext

\section{Arsitektur Infrastruktur}
\blindtext

\section{Kontainer Eksekutor}
\blindtext

\section{Orkestrasi Kontainer}
\blindtext

\section{Komunikasi Frontend-Backend}
\blindtext

