\clearpage
\chapter*{ABSTRAK}
\addcontentsline{toc}{chapter}{ABSTRAK}

%taruh abstrak bahasa indonesia di sini
\begin{center}
  \center
  \begin{singlespace}
    \bfseries \MakeUppercase{\thetitle}

    \normalfont\normalsize
    Oleh:

    \bfseries \theauthor
  \end{singlespace}
\end{center}

% \vspace{1cm}

\begin{singlespace}
  Salah satu tantangan dalam pembelajaran pemrograman bagi orang yang baru mempelajari pemrograman adalah memahami cara program bekerja. Hal ini penting karena mereka masih belum memiliki bayangan yang sesuai terhadap proses kerja program sehingga mengalami kesusahan untuk menyerap materi pemrograman yang dipelajari. Hal tersebut dipersulit apabila pembelajaran dilakukan secara daring, karena pembelajaran dapat berlangsung secara asinkron sehingga pengajar harus dapat memastikan pemahaman dapat dicapai melalui materi pembelajaran tanpa adanya interaksi secara langsung.

  Salah satu cara untuk meningkatkan pemahaman dalam pembelajaran adalah dengan membuat sistem pembelajaran interaktif (\textit{Interactive Learning Environment [ILE]}). Pada beberapa platform pembelajaran pemrograman secara daring yang sudah ada, pembelajaran interaktif dilakukan dengan pengerjaan soal menggunakan Web IDE seperti pada Sololearn, pemahaman konsep menggunakan animasi visual interaktif seperti pada Brilliant, hingga penyampaian materi yang menggunakan pembawaan narasi interaktif pada Progate. Namun, penggunaan visualisasi eksekusi program sebagai ILE masih minim digunakan pada kelas pembelajaran pemrograman, padahal menurut beberapa studi literatur hal tersebut dapat meningkatkan pemahaman konsep pemrograman karena dapat membangkitkan model konkret dalam memprogram komputer pada pelajar awam.

  Pada Tugas Akhir ini, dibangun ILE berupa kakas yang dapat memvisualisasikan eksekusi kode yang menampilkan bagaimana cara program dapat bekerja. Sistem ini diintegrasikan dengan kelas pemrograman secara daring pada Platform Web KodeBareng sehingga dapat digunakan dalam proses pembelajaran dan latihan pengerjaan soal. Setelah dilakukan eksperimen pengguna (\textit{user experiment}), didapat hasil bahwa ILE ini dinilai oleh pengguna membantu dalam pembelajarannya dengan nilai rata-rata di kisaran 4.231 dan 4.385 skala \textit{likert}, serta meningkatkan rata-rata nilai jawaban yang benar pada soal-soal yang diujikan dengan hasil peningkatan pemahaman konsep (menggunakan SOLO Level) paling signifikan pada modul 1 dengan nilai t sebesar 2.179 (\textit{p = 0.0147}) bagi orang yang belum pernah mempelajari pemrograman sebelumnya.

  \textbf{\textit{Kata kunci: pembelajaran pemrograman, pembelajaran interaktif, visualisasi eksekusi kode}}
\end{singlespace}
\clearpage
