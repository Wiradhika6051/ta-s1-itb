\chapter{Kesimpulan dan Saran}

% Bab Kesimpulan dan Saran merupakan penutup dari bagian utama Laporan Tugas Akhir. Fokuskan kesimpulan pada hal-hal baru yang relevan dengan ketercapaian tujuan Tugas Akhir terkait dengan permasalahan yang diselesaikan dalam Tugas Akhir. Saran berisi kajian hal-hal yang masih dapat dikembangkan lebih lanjut.

\section{Kesimpulan}
Setelah dilakukan analisis, implementasi, dan pengujian, dapat diambil kesimpulan sebagai berikut:
\begin{enumerate}
  \item ILE berupa kakas visualisasi eksekusi kode telah diimplementasikan pada platform web pembelajaran pemrograman KodeBareng. ILE diimplementasikan dalam 2 subsistem, yaitu pada sistem Frontend dalam bentuk komponen Visualizer serta pada sistem Autograder-Executor dalam bentuk komponen Parser dan komponen Executor menggunakan \textit{Python Debugger} (PDB) sehingga data \textit{stack trace} program dari PDB dapat diolah menjadi data JSON yang dapat divisualisasikan oleh komponen Visualizer. Komponen Visualizer tersebut telah diintegrasikan di KodeBareng pada bagian materi pembelajaran serta soal-soal kuis dan latihan kode sehingga dapat dipakai dan dieksplorasi oleh siswanya. Saat ini, ILE tersebut masih belum mengimplementasikan keseluruhan fitur yang terdapat pada bahasa pemrograman Python dan masih memiliki keterbatasan dalam visualisasi yang dihasilkan.
  \item ILE terbukti meningkatkan pengalaman belajar siswa, serta dapat meningkatkan pemahaman konsep pemrograman dan alur kerja suatu program bagi orang yang belum pernah belajar pemrograman sebelumnya. Hal ini dibuktikan dari hasil analisis pengujian pada \autoref{sec:analisis-hasil-pengujian} dengan 13 peserta grup perlakuan dan 12 peserta grup kontrol (total 25 peserta yang didapat secara \textit{convenience sampling}), peserta pada grup perlakuan merasakan dampak positif dari ILE tersebut dengan nilai 4.231 pada kriteria pemahaman terhadap kode yang dijalankan dan nilai 4.385 pada kriteria membantu menjawab kuis dan latihan soal. Selain itu, terdapat peningkatan rata-rata persentase kuis dengan jawaban yang benar secara konsisten pada, serta peningkatan rata-rata pemahaman konsep pada soal latihan kode dengan hasil paling signifikan pada modul 1 dengan nilai t sebesar 2.179 (\textit{p = 0.0147}) dengan peningkatan dari rata-rata 2.429 pada grup kontrol menjadi 3.857 rata-rata pada grup perlakuan serta pada modul 2 dengan peningkatan dari rata-rata 3.333 pada grup kontrol menjadi 3.6 rata-rata pada grup perlakuan tetapi dengan nilai t sebesar 2.262 (\textit{p = 0.662}) yang tidak signifikan akibat berkurangnya peserta eksperimen yang dapat menyelesaikan modul kedua dibandingkan modul pertama. Selain itu, ditemukan juga dampak lain dari ILE yaitu membuat distraksi selama pembelajaran berlangsung sehingga membuat waktu pembelajaran menjadi lebih lama.
\end{enumerate}

\section{Saran}
Adapun saran terkait pelaksanaan Tugas Akhir ini adalah sebagai berikut:
\begin{enumerate}
  \item ILE dapat dikembangkan lebih lanjut menggunakan skema interaksi lain, seperti penggunaan animasi dalam menampilkan visualisasi, atau elemen visual lainnya.
  \item Sejumlah peserta eksperimen mengalami kesulitan dalam berinteraksi dan navigasi pada ILE, sehingga dapat menjadi poin perbaikan untuk pengembangan selanjutnya.
  \item Dukungan terhadap fitur bahasa pemrograman Python yang lebih lengkap dan juga dukungan terhadap bahasa pemrograman lainnya. Penambahan dukungan ini memungkinkan ILE yang dikembangkan dapat digunakan pada kelas pembelajaran yang lebih kompleks.
  \item Perlu ada kajian terhadap implementasi ILE pada pendekatan lainnya seperti yang disebutkan oleh \textcite{moons2013pilot} sehingga dapat dibandingkan efektivitasnya dengan pendekatan yang dipilih pada Tugas Akhir ini.
  \item Diperlukan pengujian dengan materi pembelajaran serta latihan soal yang lebih sesuai secara akademis untuk mendapatkan hasil eksperimen yang lebih akurat. Selain itu, peserta eksperimen yang lebih banyak serta cara pengambilan sampel yang lebih merepresentasikan keseluruhan populasi juga dapat meningkatkan akurasi hasil pengujian.
\end{enumerate}