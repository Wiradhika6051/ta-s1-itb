\clearpage
\chapter*{ABSTRAK}
\addcontentsline{toc}{chapter}{ABSTRAK}

%taruh abstrak bahasa indonesia di sini
\begin{center}
  \center
  \large \bfseries \MakeUppercase{\thetitle}

  \normalfont\normalsize
  Oleh

  \theauthor
\end{center}

\vspace{1cm}

\begin{singlespace}
  Salah satu tantangan dalam pembelajaran pemrograman bagi orang yang baru mempelajari pemrograman adalah memahami cara program bekerja. Hal ini penting karena mereka masih belum memiliki bayangan yang sesuai terhadap proses kerja program sehingga mengalami kesusahan untuk menyerap materi pemrograman yang dipelajari. Hal tersebut dipersulit apabila pembelajaran dilakukan secara daring, karena pembelajaran dapat berlangsung secara asinkron sehingga pengajar harus dapat memastikan pemahaman dapat dicapai melalui materi pembelajaran tanpa adanya interaksi secara langsung.

  Salah satu cara untuk meningkatkan pemahaman dalam pembelajaran adalah dengan membuat sistem pembelajaran interaktif (\textit{Interactive Learning Environment [ILE]}). Hal ini dapat dilakukan dengan berbagai cara, seperti dengan membuat pelajar dapat berinteraksi dengan materi pembelajaran melalui visual interaktif. Pada beberapa platform pembelajaran pemrograman secara daring yang sudah ada, pembelajaran interaktif dilakukan dengan pengerjaan soal menggunakan Web IDE, pemahaman konsep menggunakan animasi visual interaktif, hingga penyampaian materi yang menggunakan konsep cerita dan menggunakan gamifikasi. Namun, penggunaan visualisasi eksekusi program masih minim digunakan pada kelas pembelajaran pemrograman, padahal menurut beberapa studi literatur hal tersebut dapat meningkatkan pemahaman konsep pemrograman karena dapat membangkitkan model kerja program komputer pada pelajar awam.

  Pada Tugas Akhir ini, dibangun ILE berupa visualisasi eksekusi kode yang dapat menampilkan bagaimana cara program dapat bekerja. Sistem ini diintegrasikan dengan kelas pemrograman secara daring pada Platform Web KodeBareng sehingga dapat digunakan dalam proses pembelajaran dan latihan pengerjaan soal. \hl{[TODO: BAHAS HASIL PENGUJIAN]}
\end{singlespace}

\textbf{\textit{Kata kunci: pembelajaran pemrograman, pembelajaran interaktif, visualisasi eksekusi kode}}
\clearpage