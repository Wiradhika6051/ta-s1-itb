\clearpage
\chapter*{ABSTRACT}
\addcontentsline{toc}{chapter}{ABSTRACT}

%taruh abstrak bahasa indonesia di sini
\begin{center}
  \center
  \begin{singlespace}
    \bfseries \MakeUppercase{Development of Interactive Learning Environment (ILE) on Programming Learning Platform KodeBareng}
  \end{singlespace}

  \begin{singlespace}
    \normalfont\normalsize
    By

    \bfseries \theauthor
  \end{singlespace}
\end{center}

% \vspace{1cm}

\begin{singlespace}
  One of the challenges in learning programming for people who are new to programming is understanding how programs work. This is important because they still don't have the right idea of how the program works, so they have difficulty absorbing the programming material they learn. This is complicated when learning is done online, because learning can take place asynchronously so that teachers must be able to ensure understanding can be achieved through learning materials without direct interaction.

  One way to improve comprehension in learning is to create an interactive learning environment (ILE). This can be done in various ways, such as by allowing learners to interact with learning materials through interactive visuals. In some existing online programming learning platforms, interactive learning is done by working on problems using a Web IDE such as Sololearn, understanding concepts using interactive visual animations such as Brilliant, to delivering material using the concept of stories and using gamification. However, the use of program execution visualization is still minimally used in programming learning classes, even though according to several literature studies it can improve understanding of programming concepts because it can evoke a working model of computer programs in ordinary students.

  In this Final Project, ILE is built in the form of a tool that can visualize code execution that can display how the program can work. This system is integrated with online programming classes on the KodeBareng Web Platform so that it can be used in the learning process and practice working on problems. After conducting user experiment, it was found that this ILE has the potential to be able to improve users' understanding of programming concepts so that it can provide a better learning experience for people who have never learned programming before.
\end{singlespace}

\textbf{\textit{Keywords: learning programming, interactive learning, code execution visualization}}
\clearpage