\clearpage
\chapter*{ABSTRACT}
% \addcontentsline{toc}{chapter}{ABSTRACT}

%taruh abstrak bahasa indonesia di sini
\begin{center}
  \center
  \large \bfseries \MakeUppercase{\thetitle}

  \normalfont\normalsize
  By

  \theauthor
\end{center}

\vspace{1cm}

\begin{singlespace}
  % \hl{[TODO: BUAT ABSTRAK BAHASA INGGRIS]}

  Abstrak berisi ringkasan apa yang telah dikerjakan dalam tugas akhir. Ada beberapa hal yang perlu diperhatikan dalam penulisan abstrak. Pertama, abstrak harus memuat permasalahan yang dikaji, metode/teknik yang digunakan untuk menyelesaikan masalah, hasil yang dicapai / evaluasi kajian, kesimpulan yang diperoleh, dan kata kunci.  Kedua, cara penulisannya harus padat dan terarah. Setiap kalimat harus dapat memberikan informasi sebanyak dan setepat mungkin, mudah dibaca dan dimengerti. Panjang ringkasan dibatasi maksimal 300 kata dan ditulis dengan satu spasi. Panjang ringkasan dibatasi maksimal 300 kata dan ditulis dengan satu spasi.

  Three points to be address in your abstract

  \begin{itemize}
    \item 1st paragraph — What are the problems in the literature/practice and why those problems need to be solved

    \item 2nd paragraph — How existing works/literature address the problems and how is the results and their open issues/challenges

    \item 3rd paragraph — How is your work address the challenges and how is the results
  \end{itemize}

\end{singlespace}

\textbf{\textit{Keywords: concise, brief, compact.}}
\clearpage