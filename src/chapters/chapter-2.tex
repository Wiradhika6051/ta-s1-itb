\chapter{Studi Literatur}

\section{Pembelajaran}
Menurut \textcite{slavin2017learn}, belajar adalah perubahan yang relatif permanen dalam perilaku atau potensi perilaku sebagai hasil dari pengalaman atau latihan yang diperkuat. Belajar merupakan akibat adanya interaksi antara stimulus dan respons. Seseorang dianggap telah belajar sesuatu jika dia dapat menunjukkan perubahan perilakunya. Menurut teori ini, dalam belajar yang penting adalah input yang berupa stimulus dan output yang berupa respons.

Teori lain mengenai pembelajaran juga dikemukakan oleh \textcite{gagne1970learning} bahwa pembelajaran adalah seperangkat peristiwa-peristiwa eksternal yang dirancang untuk mendukung beberapa proses belajar yang bersifat internal. Pembelajaran terdiri dari beberapa tipe dan setiap tipe membutuhkan instruksi yang berbeda-beda. Menurut \textcite{gagne1970learning}, terdapat 5 kategori utama dalam pembelajaran: informasi verbal, kemampuan intelektual, strategi kognitif, kemampuan motorik, serta sikap. Setiap kategori memiliki kondisi internal dan eksternal masing-masing dalam proses pembelajarannya.

Menurut \textcite{gagne1985learning}, terdapat 9 peristiwa instruksi yang dapat membantu proses pembelajaran internal pelajar, yaitu: Perhatian dan Motivasi, Pemberitahuan Objektif Pembelajaran, Stimulasi Pengulangan, Pemberian Tantangan (Stimulus), Memberikan Arahan, Keterlibatan Langsung/Pengalaman, Memberikan Saran, Penilaian Performa, serta Meningkatkan Daya Ingatan.

\section{Pemrograman}
Pada masa awal publikasi dalam pemrograman, banyak praktisi yang menjelaskan pemrograman sebagai proses yang menerjemahkan bahasa yang diketahui manusia menjadi bahasa yang dimengerti oleh mesin \parencite{mccracken1957digital}. \textcite{booth1958programming} menambahkan hubungan antara pemrograman dengan kalkulasi bahwa "Proses mengorganisasi kalkulasi dapat dibagi menjadi dua bagian -- fondasi formula matematis dan pemrograman yang sebenarnya ... menerjemahkan ... ke dalam bahasa mesin komputasi". \textcite{hartree2012calculating} menjelaskan bahwa "Proses mempersiapkan kalkulasi untuk mesin dapat dibagi menjadi 2 bagian, 'pemrograman' dan 'pengkodean'. Pemrograman adalah proses menggambarkan penjadwalan dari urutan operasi-operasi individu yang dibutuhkan untuk melakukan kalkulasi" \parencite{hartree2012calculating}.

Definisi pemrograman berubah seiring waktu dari domain matematika menjadi domain pemrosesan data yang lebih umum \parencite{hoc1990psychology}. Pada awalnya, pemrogram adalah orang yang serius dalam dunia komputer. Namun seiring dengan waktu, aplikasi-aplikasi besar banyak yang mulai menerapkan pemrograman untuk \textit{scripting} atau \textit{macro} sehingga definisi pemrograman dapat meluas hingga ke \textit{end-user} \parencite{goodell1999enduser}. Hal ini membuat \textcite{blackwell2002programming} mengusulkan bahwa pemrogram bisa menjadi siapapun yang menggunakan komputer dengan penggunaan yang berbeda-beda tergantung pada kegunaan dan kebutuhan.

\section{Pembelajaran Pemrograman secara Daring}
\blindtext

\section{\textit{Interactive Coding}}
\blindtext

\section{Arsitektur Infrastruktur}
\blindtext

\section{Kontainer Eksekutor}
\blindtext

\section{Orkestrasi Kontainer}
\blindtext

\section{Komunikasi Frontend-Backend}
\blindtext

