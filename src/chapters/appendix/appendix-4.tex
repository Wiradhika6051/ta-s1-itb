\begin{landscape}
  \chapter{Data Kuesioner Grup Kontrol} \label{appendix:data-kuesioner-kontrol}
  \scriptsize
  \begin{longtable}[c]{|l|l|l|l|l|>{\raggedright\arraybackslash\setlength{\baselineskip}{0.75\baselineskip}}p{0.12\linewidth}|>{\raggedright\arraybackslash\setlength{\baselineskip}{0.75\baselineskip}}p{0.2\linewidth}|>{\raggedright\arraybackslash\setlength{\baselineskip}{0.75\baselineskip}}p{0.2\linewidth}|}
    \hline
    \rowcolor[HTML]{C0C0C0}
    \textbf{Timestamp} & \textbf{Nama} & \textbf{Usia} & \textbf{Jenis Kelamin} & \textbf{Pendidikan} & \textbf{Konsep pemrograman apa saja yang sudah Anda pelajari sebelumnya?}                       & \textbf{Ceritakan pengalaman anda saat mengerjakan tugas yang diberikan tadi (dari segi perasaan, dsb.)}                                                                                                                                                                                                                                                                              & \textbf{Pendapat atau saran untuk pembelajaran di kodebareng}                                                                                                                                                                     \\ \hline
    \endfirsthead
    %
    \multicolumn{8}{c}%
    {{\bfseries Tabel \thetable\ dilanjutkan dari halaman sebelumnya}}                                                                                                                                                                                                                                                                                                                                                                                                                                                                                                                                                                                                                                                                                                                                                              \\
    \hline
    \rowcolor[HTML]{C0C0C0}
    \textbf{Timestamp} & \textbf{Nama} & \textbf{Usia} & \textbf{Jenis Kelamin} & \textbf{Pendidikan} & \textbf{Konsep pemrograman apa saja yang sudah Anda pelajari sebelumnya?}                       & \textbf{Ceritakan pengalaman anda saat mengerjakan tugas yang diberikan tadi (dari segi perasaan, dsb.)}                                                                                                                                                                                                                                                                              & \textbf{Pendapat atau saran untuk pembelajaran di kodebareng}                                                                                                                                                                     \\ \hline
    \endhead
    %
    5/30/2022 11:03:18 & N             & 17            & Laki-laki              & SMA/SMK             & Input/Output                                                                                    & Untuk pembelajarannya, sangat menarik, selain dalam bentuk paragraf, Kodebareng juga memberikan contoh contoh yang dapat membuat saya lebih paham lagi. Saya pula merasa sangat tertantang ketika diberikan soal soal quiz di akhir pembelajaran. Kesalahan kesalahan yang saya perbuat lebih mengajarkan saya untuk lebih mengerti lagi akan pemahaman modul modul sebelumnya        & Menurut saya webnya sudah sangat bagus untuk pemula yang ingin belajar.                                                                                                                                                           \\ \hline
    5/30/2022 11:05:25 & O             & 17            & Laki-laki              & SMA/SMK             & Input/Output, Tipe Data, Kondisional (If-Else), Pengulangan (Looping), Fungsi (Function/Method) & Bisa membaca penjelasan materi dengan fokus, kadang kalo baca tulisan banyak kyk gitu udh males liatnya, tpi tdi biasa aja, malah jdi pengen baca                                                                                                                                                                                                                                     & Sudah baguss, mungkin ke depannya bisa ditambah fitur untuk mencoba sambil belajar, bukan hanya kuis di akhir, tetapi juga sambil belajar materi sambil coba ngodingnya gt. Tpi overall konsep dr materinya dapet/bisa dimengerti \\ \hline
    5/30/2022 11:25:37 & P             & 19            & Laki-laki              & SMA/SMK             & Kondisional (If-Else), Pengulangan (Looping)                                                    & merasa puas saat berhasil memecahkan case yang ada walau masi sederhana tetapi saya menemui kendala saat saya mencoba kode di code editor vscode.                                                                                                                                                                                                                                     & saya rasa udah bagus dari UI dan lain lainnya.Webnya  enak untuk diliat dan easy to use juga                                                                                                                                      \\ \hline
    5/30/2022 11:30:44 & Q             & 17            & Laki-laki              & SMA/SMK             & Input/Output                                                                                    & Pada awalnya, saya cukup gugup karena tidak pernah menjalankan pembelajaran pemrograman dalam ruang google meeting seperti ini. Akan tetapi, proses pembelajarannya nyaman dan beberapa kebingungan dan pertanyaan yang saya dapat, bisa diperjelas dalam pembelajaran interaktif ini. Saya belajar banyak dari kelas ini dan jadi lebih senang belajar pemrograman karena kelas ini. & Menurut saya, quiz dan pertanyaan setelah pelajaran mungkin dapat ditambah agar dapat memastikan kita paham mengenai materinya dengan jelas. Selain dari itu sudah cukup baik.                                                    \\ \hline
    5/31/2022 10:19:23 & R             & 20            & Perempuan              & Sarjana             & Input/Output, Tipe Data, Kondisional (If-Else), Pengulangan (Looping), Fungsi (Function/Method) & Perasaannya lebih ke menarik dan penasaran sih, nge-refresh kembali materi waktu TPB.                                                                                                                                                                                                                                                                                                 & ditambahkan media coding di materi (jadi ga cuma di latihan) supaya bisa dicoba-coba dari contoh yang diberikan.                                                                                                                  \\ \hline
    5/31/2022 10:25:59 & S             & 21            & Laki-laki              & Sarjana             &                                                                                                 & saya merasa senang, karena saya merupakan orang yang tidak mengenal pemograman sehingga dengan mengerjakan tugas dari Kode Bareng saya mengenal dan mengetahui basic pemograman                                                                                                                                                                                                       & saran saya untuk KodeBareng adalah ditambahkan video pembelajaran supaya lebih menarik lagi                                                                                                                                       \\ \hline
    5/31/2022 13:48:53 & T             & 18            & Laki-laki              & Sarjana             & Input/Output, Tipe Data, Kondisional (If-Else)                                                  & Seru sekali, biasanya kalau belajar hanya dari youtube tidak konsisten, kadang terdistraksi, tapi kalau pakai kode bareng tidak mudah terdistraksi dan bisa jauh lebih fokus serta bisa menerap lebih cepat.                                                                                                                                                                          & Tidak ada, sudah sangat bagus, semoga kedepannya saya tetap bisa menggunakan kode bareng :D                                                                                                                                       \\ \hline
    5/31/2022 19:44:13 & U             & 18            & Perempuan              & SMA/SMK             & Input/Output, Tipe Data, Kondisional (If-Else), Pengulangan (Looping)                           & Metode online learning website kodebareng ini mempermudah saya untuk mempelajari pemrograman. Didukung dengan mini quiz di akhir setiap modul yang dapat menjadi tolak ukur kemampuan saya dalam memahami materi. Saya merasa puas saat dapat mengerjakan soal dengan benar.                                                                                                          & Fontnya kayaknya bisa dibesarin sedikit kak                                                                                                                                                                                       \\ \hline
    6/1/2022 11:33:42  & V             & 18            & Laki-laki              & SMA/SMK             &                                                                                                 & seru dan menantang. Sehabis ini saya mau langsung belajar phyton                                                                                                                                                                                                                                                                                                                      & Tidak ada, UI nya enak dan materinya mudah dipahami                                                                                                                                                                               \\ \hline
    6/1/2022 11:34:44  & W             & 18            & Laki-laki              & SMA/SMK             & Input/Output, Tipe Data                                                                         & dari awal saya memang antusias belajar phyton, lalu saya dikenalin kode bareng sama ka mario. menurut saya untuk user experience nya cukup simple, asik dan to the point                                                                                                                                                                                                              & saran saya buat fitur untuk mempermudah melihat page sebelumnya                                                                                                                                                                   \\ \hline
    6/1/2022 14:14:30  & X             & 18            & Perempuan              & SMA/SMK             &                                                                                                 & Seru, bisa nambah ilmi buat akt yang bener-bener awam dan gangerto sm jenis2 pemograman karena gapernah belajar atau cari tahu sebelumnya. Buat pandan belajar di KodeBareng juga menarik until tampilannya.                                                                                                                                                                          & sarannya mungkin bisa ditambah video di webnya mangenai cara2nya sopaya bisa lebih mudah dimengerti, tapi sejauh ini untuk tampilannya sudah bagus!                                                                               \\ \hline
    6/1/2022 16:46:03  & Y             & 18            & Laki-laki              & Sarjana             & Input/Output, Tipe Data                                                                         & Tugas yang diberikan cukup sederhana. Cocok untuk mengetes pengetahuan pemrograman bagi pemula                                                                                                                                                                                                                                                                                        & Media pembelajaran di kodebareng sangat bagus dan keren. visualisasinya juga lumayan bagus. Mungkin tata letak ketika kuis yang disuruh untuk menjalankan input sedikit dibenahi                                                  \\ \hline
  \end{longtable}
\end{landscape}