\clearpage
\chapter*{ABSTRACT}
\addcontentsline{toc}{chapter}{ABSTRACT}

%taruh abstrak bahasa indonesia di sini
\begin{center}
  \center
  \large \bfseries \MakeUppercase{Development of Interactive Learning Environment (ILE) on Programming Learning Platform KodeBareng}

  \normalfont\normalsize
  By

  \theauthor
\end{center}

\vspace{1cm}

\begin{singlespace}
  One of the challenges in learning programming for people who are just learning is understanding how programs work. This is important because they do not yet have an image that is in accordance with the program process so that it is difficult for them to absorb the material being studied. This is complicated if the learning is done boldly, because learning can take place asynchronously so the teacher must ensure that understanding can be achieved through learning without any direct interaction.

  One way to improve understanding in learning is to create an interactive learning system (ILE). This can be done in various ways, such as by enabling students to interact with learning materials through interactive visuals. On several existing online programming learning platforms, interactive learning is carried out by working on questions using a Web IDE such as Sololearn, understanding concepts using interactive visual animation as in Brilliant, to delivering material using story concepts and using gamification. However, the use of visualization of program execution is still minimally used in programming learning classes, even though according to several literature studies it can improve understanding of programming concepts because it can generate working models of computer programs in lay students.

  In this Final Project, ILE is built in the form of a tool that can visualize code execution that can show how the program can work. This system is integrated with online programming classes on the KodeBareng Web Platform so that it can be used in the learning process and practice questions. After conducting user experiments, it was found that ILE has the potential to increase user understanding of programming concepts so that it can provide a better learning experience for people who have never studied programming before.
\end{singlespace}

\textbf{\textit{Keywords: learning programming, interactive learning, code execution visualization}}
\clearpage