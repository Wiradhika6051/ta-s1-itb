\chapter{Kesimpulan dan Saran}

% Bab Kesimpulan dan Saran merupakan penutup dari bagian utama Laporan Tugas Akhir. Fokuskan kesimpulan pada hal-hal baru yang relevan dengan ketercapaian tujuan Tugas Akhir terkait dengan permasalahan yang diselesaikan dalam Tugas Akhir. Saran berisi kajian hal-hal yang masih dapat dikembangkan lebih lanjut.

\section{Kesimpulan}
Setelah dilakukan analisis, implementasi, dan pengujian, dapat diambil kesimpulan sebagai berikut:
\begin{enumerate}
  \item Telah berhasil dikembangkan ILE berupa kakas visualisasi eksekusi kode yang telah diintegrasikan pada platform web pembelajaran KodeBareng. ILE tersebut telah diintegrasikan pada modul materi pembelajaran serta soal-soal kuis dan latihan kode sehingga dapat dipakai dan dieksplorasi oleh penggunanya. Saat ini, ILE tersebut masih belum mengimplementasikan keseluruhan fitur yang terdapat pada bahasa pemrograman Python dan masih memiliki keterbatasan dalam visualisasi yang dihasilkan. Keunggulan dari ILE tersebut adalah belum ada platform web pembelajaran pemrograman lain yang memiliki kakas visualisasi eksekusi kode yang terintegrasi pada kelas pembelajarannya sehingga dapat menjadi keunggulan dari platform web pembelajaran KodeBareng.
  \item ILE yang telah dibuat memiliki dampak terhadap pengalaman pembelajaran penggunanya, serta memiliki indikasi dapat meningkatkan pemahaman konsep pemrograman dan alur kerja suatu program bagi orang yang belum pernah belajar pemrograman sebelumnya sehingga dapat lebih mudah mempelajari pengenalan dunia pemrograman.
  \item
\end{enumerate}

\section{Saran}
Adapun saran terkait pelaksanaan Tugas Akhir ini adalah sebagai berikut:
\begin{enumerate}
  \item Desain tampilan dan interaksi pada komponen visualisasi pada ILE yang telah dikembangkan masih belum mempertimbangkan aspek-aspek psikologis. Dengan mempertimbangkan aspek tersebut, visualisasi yang dibuat akan semakin mudah dimengerti oleh pengguna dan lebih nyaman dalam memakai ILE.
  \item Dukungan terhadap fitur bahasa pemrograman Python yang lebih komprehensif. Penambahan dukungan ini memungkinkan ILE yang dikembangkan dapat digunakan pada kelas pembelajaran yang lebih kompleks.
  \item Konsultasi terhadap tenaga kerja ahli dalam pembuatan materi pembelajaran serta penilaian jawaban sehingga hasil eksperimen yang didapat lebih akurat dan pembelajaran lebih sesuai dengan kaidah-kaidah pendidikan.
\end{enumerate}