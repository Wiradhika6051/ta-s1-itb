\chapter{Rencana Penyelesaian Masalah}

Pada bab ini dituliskan analisis mengenai masalah yang diangkat di tugas akhir ini. Analisis mencakup analisis masalah yang telah dirumuskan serta solusi-solusi yang dapat diterapkan untuk menyelesaikan masalah-masalah tersebut.

\section{Analisis Masalah}
Saat ini, terdapat banyak sekali metode penyampaian materi pembelajaran pemrograman secara daring. Namun, agar dapat memvalidasi pengetahuan yang didapat diperlukan sistem pendukung yang dapat digunakan sebagai media praktik bagi para pelajar. Kurangnya praktik dan latihan dalam pembelajaran dapat membuat adanya pemisah antara pemahaman teori dan praktik. (\textit{!TODO: ceritakan paper mengenai ini})

Untuk memvalidasi pengetahuan yang sudah dipelajari, terdapat banyak metode yang dapat digunakan. Salah satu metode yang sering dipakai adalah kuis yaitu serangkaian pertanyaan yang mengacu pada materi yang telah diberikan. Kuis dapat dibuat dalam berbagai macam bentuk seperti pilihan ganda, isian singkat, esai, dll. Metode esai dapat digunakan untuk memvalidasi logika dan pola pikir dari pemecahan masalah, namun karena esai tersebut merupakan kode maka harus terdapat sistem yang dapat mengeksekusi, menilai, serta memberikan \textit{feedback} dari hasil eksekusi kode tersebut layaknya pemrograman yang sebenarnya.

Maka dari itu, diperlukan sistem pembelajaran pemrograman interaktif yang dapat dipakai pengguna untuk menuliskan kode, memberikan \textit{feedback} dari kode yang dibuat, serta menilai kebenaran dari kode tersebut. \textit{Feedback} yang diberikan berupa hasil eksekusi kode berupa pesan keluaran apabila kode berhasil dijalankan, serta pesan error apabila terjadi masalah dalam eksekusi kode. Pesan keluaran hasil eksekusi dapat dibandingkan dengan pesan keluaran yang seharusnya agar dapat dinilai dan diberitahukan kepada pengguna sehingga pengguna dapat mengetahui letak kesalahan dari kode yang diimplementasikan.

\section{Analisis Solusi}

