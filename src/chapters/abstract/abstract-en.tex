\clearpage
\chapter*{ABSTRACT}
\addcontentsline{toc}{chapter}{ABSTRACT}

%taruh abstrak bahasa indonesia di sini
\begin{center}
  \center
  \begin{singlespace}
    \bfseries \MakeUppercase{Development of Interactive Learning Environment (ILE) on Programming Learning Platform KodeBareng}

    \normalfont\normalsize
    By

    \bfseries \theauthor
  \end{singlespace}
\end{center}

% \vspace{1cm}

\begin{singlespace}
  One of the challenges in learning programming for people who are new to programming is understanding how programs work. This is important because they still don't have an appropriate view of the program's working process, making it difficult for them to absorb the programming material they learn. This is made more difficult when learning is done online, because learning can take place asynchronously so that teachers must be able to ensure understanding can be achieved through learning materials without direct interaction.

  One way to improve understanding in learning is by creating an interactive learning system (ILE). In some existing online programming learning platforms, interactive learning is done by working on problems using a Web IDE such as Sololearn, understanding concepts using interactive visual animations such as Brilliant, and presenting material using interactive storytelling on Progate. However, the use of visualization of the execution of the program as an ILE is still minimally used in programming learning classes, even though based on several literature studies it can improve the understanding of programming concepts because it can generate a concrete model of computer programming in novice students.

  In this Final Project, an ILE is built in the form of a code execution visualization tool that displays how the program works. This system is integrated with online programming classes on KodeBareng Web Platform so that it can be used in the learning process and practice problem solving. After conducting user experiments, the results show that this ILE is considered by users to be helpful in learning with an average score in the range of 4.231 and 4.385 on the likert scale, as well as increasing the average correct answer value on the questions tested with the most significant increase in concept understanding (using SOLO Level) in module 1 with a t value of 2.179 (\textit{p = 0.0147}) for students who have never learned programming before with an increase from an average SOLO Level score of 2.429 to 3.857 and in module 2 from an average SOLO Level of 3.333 to 3.6.

  \textbf{\textit{Keywords: learning programming, interactive learning, code execution visualization}}
\end{singlespace}

\clearpage
